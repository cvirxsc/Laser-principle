\documentclass[utf8]{ctexart} %中文文档类
\usepackage[a4paper,left=2.50cm, right=2.50cm, top=2.50cm, bottom=2.50cm]{geometry} % 页边距
\usepackage{indentfirst} % 首行缩进
\usepackage{fancyhdr} % 设置页眉、页脚
\renewcommand\headrulewidth{0pt}% 页眉与正文之间的水平线粗细
\usepackage{ctexcap} % 标题是中文的
\usepackage{helvet} % 用来指定beamer使用的字体
\usepackage{hyperref} % bookmarks 
\usepackage{multicol} % 分栏

\usepackage{amsmath, amsfonts, amssymb} % 数学公式、符号
\usepackage[english]{babel} % 数学公式标准
\usepackage{bm} % 加粗方程字体 

\usepackage{graphicx} % 图片 
\usepackage{url} % 超链接 

\usepackage{multirow} % 表格
\usepackage{booktabs} % 三线表
\usepackage{longtable} % 长表格

\usepackage{algorithm} % 算法或伪代码
\usepackage{algorithmic} % 算法或伪代码

\usepackage{enumitem} % 枚举环境宏包
\usepackage{listings}
\usepackage{xcolor}
\renewcommand{\algorithmicrequire}{ \textbf{Input:}} 
\renewcommand{\algorithmicensure}{ \textbf{Initialize:}} 
\renewcommand{\algorithmicreturn}{ \textbf{Output:}} %算法格式

\pagestyle{fancy} \lhead{} \chead{} \lfoot{} \cfoot{} \rfoot{}
\pagestyle{plain}
\hypersetup{colorlinks, bookmarks, unicode} % unicode 
 
 \lstset{
% basicstyle=\footnotesize,                 % 设置整体的字体大小
showstringspaces=false,                     % 不显示字符串中的空格
frame=single,                               % 设置代码块边框
numbers=left,                               % 在左侧显示行号
% numberstyle=\footnotesize\color{gray},    % 设置行号格式
numberstyle=\color{darkgray},               % 设置行号格式
backgroundcolor=\color{white},              % 设置背景颜色
keywordstyle=\color{blue},                  % 设置关键字颜色
commentstyle=\it\color[RGB]{0,100,0},       % 设置代码注释的格式
stringstyle=\sl\color{red},                 % 设置字符串格式
}

\title{\textbf{激光原理作业}}
\author{\bf 200212127席思诚}
\date{\today}

\hypersetup{
colorlinks=true,
linkcolor=black
} % 使目录为黑色字
\begin{document}  
		\maketitle
		
        \section{问题一的求解}
        作业2-1:利用 matlab 画出高阶拉盖尔﹣高斯光束和高阶厄米﹣高斯光束。
        \begin{figure}[htbp]
            \centering
            \includegraphics[width=10cm]{a.png}
            \caption{Caption}
            \label{fig:my_label}
        \end{figure}
        \section{问题二的求解}
        作业2-2:根据高斯光束特性,利用 matlab 分别画出束腰半径为1mm的高斯光束在束腰处的二维和三维光强分布图。
        \newpage
        \section*{附录}

        \begin{lstlisting}[language=matlab]
        
    m=0;             % Beam order
    n=0;                % Beam order
    count = 0;
    w0 = 2.0;               % Beam waist
    k = 2*pi/532.0e-9;      % Wavenumber of light

    zR = k*w0^2/2;      % Calculate the Rayleigh range

    % Setup the cartesian grid for the plot at plane z
    z = 0.0;
    [xx, yy] = meshgrid(linspace(-5, 5), linspace(-5, 5));

    U00 = 1/(1 + 1i*z/zR) .* exp(-(xx.^2 + yy.^2)/w0^2./(1 + 1i*z/zR));
 
    for m=0:3
        for n=0:3
            count=count+1;
            Hn = hermiteH(n, xx);
            Hm = hermiteH(m, yy);
            U = U00.*Hn.*Hm.*exp(-1i*(n + m).*atan(z/zR));
            subplot(4,4,count)
            imagesc(abs(U).^2);
            title([num2str(n),num2str(m)]);
        end
    end
    
   
    

        \end{lstlisting}


\end{document}